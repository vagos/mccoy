\usepackage{datetime}
\usepackage{wrapfig}
\usepackage{enumerate}
\usepackage{float}
\usepackage{graphicx}
\usepackage{caption}
\usepackage[acronym]{glossaries}
\usepackage{glossaries-polyglossia}
\usepackage[nottoc]{tocbibind}
\usepackage{siunitx}
\usepackage{subcaption}
\usepackage{geometry}
\usepackage{fancyhdr}
\usepackage{cite}
\usepackage{subfiles}
\usepackage{amsmath}
\usepackage{array,tabularx}
\usepackage{chngcntr}
\usepackage{afterpage}
\usepackage{ulem}
\usepackage{polyglossia}
\usepackage{fontspec}
\usepackage{hyperref}
\usepackage{dirtytalk}
\usepackage{algorithm2e}
\usepackage{enumitem,amssymb}
\usepackage{pifont}
\usepackage{amsmath}
\usepackage{listings}
\usepackage{xcolor}
\usepackage{formal-grammar}
\usepackage{varwidth}
\usepackage{fmtcount}
\usepackage{tikz}
\usepackage{tikzpeople}
\usepackage{array}
\usepackage[framemethod=tikz]{mdframed} % Allows defining custom boxed/framed environments
\usepackage{cleveref}
\usepackage{epigraph}
\usepackage{microtype}
\usepackage{dirtytalk}

\usetikzlibrary{trees}
\usetikzlibrary{fit}
\usetikzlibrary{shapes}
\usetikzlibrary{arrows.meta, positioning, shadows}
\usetikzlibrary{chains,shapes.multipart}
\usetikzlibrary{external}
\usetikzlibrary{matrix}

\tikzstyle{rect} = [rectangle,fill=white, text centered]
\tikzstyle{database} = [draw, cylinder, shape border rotate = 90, aspect = 0.2]
\tikzstyle{thick-arrow} = [->, thick]
\tikzstyle{arrow} = [thick,->,>=stealth]
\tikzstyle{arrow-small} = [->,>=stealth]
\tikzstyle{simple-rect} = [rectangle, text centered, draw = black, inner sep=4mm]

\tikzset{
    doc/.style={draw, minimum height=4em, minimum width=3em, 
                fill=white, 
                double copy shadow={shadow xshift=4pt, 
                             shadow yshift=4pt, fill=white, draw}}
}

\tikzset{
    dcs/.style = {double copy shadow},
}

\setmainfont[Mapping=tex-text]{CMU Serif}
\setsansfont[Mapping=tex-text]{CMU Sans Serif}
\setmonofont{CMU Typewriter Text}

\setotherlanguage[variant=british]{english}
\setmainlanguage[]{greek}

\hypersetup{
    colorlinks=true,
    filecolor=magenta,      
    urlcolor=blue,
}

\hypersetup{linkcolor=black}

\newcommand{\src}[1]{\texttt{#1}}

\lstset{basicstyle=\footnotesize\ttfamily,breaklines=true, captionpos=b}

% MD Frames

\mdfdefinestyle{commandline}{
	leftmargin=10pt,
	rightmargin=10pt,
	innerleftmargin=15pt,
	middlelinecolor=black!50!white,
	middlelinewidth=2pt,
	frametitlerule=false,
	backgroundcolor=black!5!white,
	frametitle={LLM Output},
	frametitlefont={\normalfont\sffamily\color{white}\hspace{-1em}},
	frametitlebackgroundcolor=black!50!white,
	nobreak,
	singleextra={%
    }
}

% Define a custom environment for command-line snapshots
\newenvironment{commandline}{
	\medskip
	\begin{mdframed}[style=commandline]
}{
	\end{mdframed}
	\medskip
}

\mdfdefinestyle{warning}{
	topline=false, bottomline=false,
	leftline=false, rightline=false,
	nobreak,
	singleextra={%
		\draw(P-|O)++(-0.5em,0)node(tmp1){};
		\draw(P-|O)++(0.5em,0)node(tmp2){};
		\fill[black,rotate around={45:(P-|O)}](tmp1)rectangle(tmp2);
		\node at(P-|O){\color{white}\scriptsize\bf ?};
		\draw[very thick](P-|O)++(0,-1em)--(O);%--(O-|P);
	}
}

% Define a custom environment for warning text
\newenvironment{prompt}[1][Prompt:]{ % Set the default warning to "Warning:"
	\medskip
	\begin{mdframed}[style=warning]
		\noindent{\textbf{#1}}
}{
	\end{mdframed}
}

\renewcommand{\lstlistingname}{Καταχώριση}